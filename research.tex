\documentclass[a4paper,11pt]{article}

\usepackage[top=2.5cm, bottom=2.5cm, left=2.5cm, right=2.5cm]{geometry}

\usepackage{fancyhdr}
\pagestyle{fancy}
\fancyhf{}
\rhead{\thepage}

\usepackage{times}

% add paragraph breaks
\setlength{\parskip}{0.5em}
\renewcommand{\baselinestretch}{1.0}
\setlength{\parindent}{0em}

\title{Side Channel Attacks in the Browser}
\date{}
 
\begin{document}

\maketitle

\section*{Case for Support}

a six-page (maximum) Case for Support: an account of the motivation for the work, the approach being taken, and a breakdown of the work to be carried out (often in terms of "workpackages"), plus a management plan (how will the project be managed such that it delivers what you have promised to do) and any other information that usefully describes the nature of the project being proposed)

\newpage
\section*{Budget}

a one-page (maximum) Budget: a table/spreadsheet of expenditure

\newpage
\section*{Justification For Resources}

a one-page (maximum) Justification For Resources: essentially a written narrative on why you need the expenditure on each line-item in your budget

\newpage
\section*{Impact Statement}

a one-page (maximum) Impact Statement: a description of how you intend to ensure that your work makes a difference in the world, rather than sitting on a shelf gathering dust)

\newpage
\section*{Workplan}

and also a one-page (maximum) Workplan:typically a GANTT chart or similar diagram indicating the order in which workpackages are carried out)



\bibliographystyle{ieeetr}
\bibliography{research.bib}

\end{document}