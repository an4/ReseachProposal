\documentclass[a4paper,11pt]{article}

\usepackage[top=2.5cm, bottom=2.5cm, left=2.5cm, right=2.5cm]{geometry}

\usepackage{fancyhdr}
\pagestyle{fancy}
\fancyhf{}
\rhead{\thepage}

\usepackage{times}

% add paragraph breaks
\setlength{\parskip}{0.5em}
\renewcommand{\baselinestretch}{1.0}
\setlength{\parindent}{0em}

\title{Side Channel Attacks in the Browser}
\date{}
 
\begin{document}

\maketitle

\section*{Case for Support}

%a six-page (maximum) Case for Support: an account of the motivation for the work, the approach being taken, and a breakdown of the work to be carried out (often in terms of "workpackages"), plus a management plan (how will the project be managed such that it delivers what you have promised to do) and any other information that usefully describes the nature of the project being proposed)

\subsection*{Background}
Side-Channel Attacks (SCA) are an important class of cryptanalytic techniques which target unintentional information leakage in order to disclose private data. In general, Side-Channel attacks monitor a device performing an interesting operation and try to find a correlation between the side-channel information and the internal state of the processing device, which is related to the secret parameters involved in the computation. Side-channel information can be obtained by monitoring variations in the execution time (\textit{Timing Attacks}), the power consumption (\textit{Power Analysis}), the electro-magnetic emission (\textit{Electro-magnetic Analysis - EMA}), or the state of the cache memory (\textit{Cache Attacks}) \cite{standaert2010introduction}. The goal of an attacker in side-channel attacks is to either recover the secret parameters involved in the computation by targeting faults in the implementation of some cryptographic algorithm, or disclose private information about a user, such as their location or browsing history. Side-channel attacks exploit vulnerabilities in a specific implementation rather than targeting an abstract algorithm, making them less generic but more powerful \cite{standaert2010introduction}.

The expansion of the Internet determined browser vendors to add more and more features to their software in order to offer richer, dynamic and more interactive web pages. The evolution of browser software not only improved the overall user experience but also opened the way for a new category of side-channel attacks known as browser attacks. These types of attacks target vulnerabilities in the browser software. Browser attack fall into two categories: traditional timing attacks which measure the time it takes the server to respond, and cache timing attacks which measure how long it takes to retrieve a resource from the browser's cache memory.

Browser attacks are a very powerful category of side-channel attacks, which require minimal set up and affect a wide range of users. In general, these types of attacks require a user to navigate to an untrusted website, which contains attacker-controlled content. The web page has a background script which collects information about the victim which can be used by the attacker to infer a user's location, browsing history, or even current browsing activity. The victims are usually unaware that the website is collecting private data about them. One might think that such attacks can be avoided by simply not visiting untrusted websites; however, researchers discovered that even high-profile websites are using similar techniques to collect information about their visitors\cite{jang2010empirical}. For example, a health insurance provider might want to know if the user has previously visited any health related web pages, this can easily be achieved using a side-channel attack.

Vulnerabilities in browser software emerge every day, but they also disappear within weeks. This gives attackers a limited time frame in which they can exploit potential vulnerabilities in web browsers. Although, most browser vulnerabilities get patched immediately, there are side-channel attacks in the browser discovered over fifteen years ago which are still possible. 

In 2000, Felten and Schneider\cite{felten2000timing} presented the first browser side-channel attack, where they were able to determine if the user has previously visited a web page. A few years later, Bortz et al.\cite{bortz2007exposing} used a similar technique to determine the size of a hidden file, which can disclose private information about the user. Their work was extended by Van Goethem et al.\cite{van2015clock}, who used newer web development technologies in order to determine the size of a remote origin resource. Browser attacks are not only used to determine how much information a user has access to, in 2015 Jia et al.\cite{jia2015know} demonstrated that it is possible to determine a user's location. Additionally, Oren et al.\cite{oren2015spy} showed that it is possible to disclose a user's current browsing activity by analysing the state of their browser's cache memory.

\subsection*{Motivation}

The aforementioned papers present various techniques for disclosing private information about users, by either measuring the amount of data the users have access to, or simply determining what pages they previously accessed by checking the state of their browser's cache memory. Additionally, most of the papers present real world scenarios where the attacks can be used to de-anonymize a user; however, the attacks are presented from the perspective of a malicious user who is trying to gain access to private data.

The world of browser attacks is rapidly changing with new vulnerabilities being discovered and patched every week. In general, adversaries have a limited time frame in which they can exploit the newly found vulnerabilities. In general, vulnerabilities found in the browser software are fixed within weeks; however, sometimes the measures proposed for stopping the attacks will bring severe performance penalties and browser vendors refuse to include them in their software, or a countermeasure is yet to be found. This is the case of the attacks discovered by Felten et al.\cite{felten2000timing} and Bortz et al.\cite{bortz2007exposing}, which have been around for years. 

This motivated us to investigate new applications for this particular kind of browser attacks. Our research will explore how side-channel information can be used to obtain a user's list of connections on social networking websites. In their paper, Van Goethem et al.\cite{van2015clock}, briefly mention a real world application of the browser attack in which they can determine if two people are friends on Facebook, or connected on LinkedIn. We are going to use their idea and try to efficiently determine a user's full list of contacts on any social media platform.


\subsection*{Project Objectives}

The aim of this research is to provide theory, tools and techniques for disclosing a user's list of connections on social media platforms. More specifically, we will focus on the following objectives:
\begin{enumerate}
\item To analyse all the existing browser attacks which can be used to determine if two users are friends and determine which technique performs the best.
\item To build a basic model which determines a user's connections on social media websites. 
\item To test the model on data provided by security companies. This will allow us to improve the algorithms.
\item To test the model in the real world. We will try to determine a user's list of connections on real social medial website.
\end{enumerate}

\subsection*{Programme and Methodology}
Personnel listed in the program of work are as follows:\\
Principal Investigator: Prof. Freddie Mercury (FM), University of Bristol\\
Co-Investigator: Dr. Brian May (BM), University of Bristol\\
Co-Investigator: Dr. Roger Taylor (RT), University of Bristol - Law thing - data \\
Post-doctoral Research Associate (PDRA1)\\
Post-doctoral Research Associate (PDRA2)\\

\subsubsection*{Workpackage 1: A preliminary analysis of published attack vectors}


% they affect lots of users
% how to use them


% example for security agency
% motivate why you want to research ways about disclosing a user's private network






% maybe par that they are still possible, affect lots of users (basically summarize the previous page and say what I want to do in my research) 
% and say that none of them looked into the practical side, they merely researched new methods for ethical reasons

\bibliographystyle{unsrt}
\bibliography{research}

\newpage
\section*{Budget}

a one-page (maximum) Budget: a table/spreadsheet of expenditure

\newpage
\section*{Justification For Resources}

a one-page (maximum) Justification For Resources: essentially a written narrative on why you need the expenditure on each line-item in your budget

\newpage
\section*{Impact Statement}

a one-page (maximum) Impact Statement: a description of how you intend to ensure that your work makes a difference in the world, rather than sitting on a shelf gathering dust)

\newpage
\section*{Workplan}

and also a one-page (maximum) Workplan:typically a GANTT chart or similar diagram indicating the order in which workpackages are carried out)

\end{document}